\documentclass[12pt,a4paper]{article}

\usepackage[utf8]{inputenc}
\usepackage{mathptmx}
\usepackage[T1]{fontenc}
\usepackage[margin=1in]{geometry}
\usepackage{lipsum} % For placeholder text, remove if not needed

\begin{document}

\section*{Sommaire Exécutif}

\hrule
\vspace{0.4cm}

\textbf{Titre:} Projet Recherche - Emulation d'APT par Caldera et Analyse. \\
\textbf{Participants:} Ewen GOURVES, Aymeric ROBIN, Paul SCHELLER.

\section*{Description}
Ce projet de recherche vise à analyser les attaques de type APT (Advanced Persistent Threat) en utilisant l'outil Caldera.
L'objectif principal est de simuler des scénarios d'attaques et d'évaluer la possibilité de les détecter et les associés à des techniques connues.
L'outil Caldera permet de simuler des attaques en utilisant des techniques du cadre MITRE ATT\&CK, ce qui facilite l'analyse et la compréhension des comportements malveillants.
Aurora EDR est utilisé pour la détection des attaques, il permet notamment d'identifier des techniques d'attaques et de les corréler avec des événements de sécurité.

\subsection*{Objectifs clés}
\begin{itemize}
    \item Simuler des attaques APT en utilisant Caldera.
    \item Analyser les résultats obtenus par Aurora EDR.
    \item Évaluer la capacité de détection d'Aurora EDR face aux attaques simulées.
    \item Identifier les techniques d'attaques utilisées.
\end{itemize}

\subsection*{Résultats}
\begin{itemize}
    \item L'outil Caldera a permis de simuler plusieurs scénarios d'attaques APT.
    \item Les attaques simulées ont été détectées par Aurora EDR. Certaines ont pu être identifiées à des techniques du cadre MITRE ATT\&CK.
    \item Il a été possible d'associer les techniques reconnues à des APT connues. Cependant, nous n'avons pas pu établir le lien entre les techniques détectées et l'APT simulé (Akira).
\end{itemize}

\subsection*{Conclusion}
\begin{itemize}
    \item L'outil Caldera est efficace pour simuler des attaques APT et automatiser ce processus.
    \item Aurora EDR est capable de détecter ces attaques et de les corréler avec des techniques connues.
    \item Cependant, il manque de règles de détection pour certaines techniques, ce qui limite son efficacité.
    \item Le rapprochement entre les techniques reconnues et les APT est une tâche difficile. En particulier, trouver la bonne méthode pour regrouper les techniques détectées est un défi.
\end{itemize}

\end{document}